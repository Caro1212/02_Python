 \documentclass[12pt]{article}
 % LaTeX file created by GUIDE version 39.0 on 02/18/22 at 17:24
 %%%% Usage: latex tree_ts.tex, dvips tree_ts.tex, ps2pdf tree_ts.tex
 %%%% or: xelatex tree_ts.tex
 \usepackage{pstricks}
 %%%% If using pdflatex, replace above line with \usepackage[pdf]{pstricks}
 %%%% and use: pdflatex -shell-escape tree_ts.tex
 \usepackage{pstricks,pst-node,pst-tree}
 \usepackage{geometry}
 \usepackage{lscape}
 \definecolor{wheat}{rgb}{0.96, 0.87, 0.7}
 \definecolor{lightgray}{rgb}{0.9, 0.9, 0.9}
 \definecolor{aqua}{rgb}{0.0, 1.0, 1.0}
 \definecolor{mauve}{rgb}{0.88, 0.69, 1.0}
 \definecolor{skyblue}{RGB}{86,180,233}
 \definecolor{vermillion}{RGB}{213,94,0}
 \definecolor{purple}{RGB}{204,121,167}
 \definecolor{bluishgreen}{RGB}{0,158,115}
 \pagestyle{empty}
 \begin{document}
 %\begin{landscape}
 \begin{center}
\psset{linecolor=black,tnsep=1pt,tndepth=0cm,tnheight=0cm,treesep=1.5cm,levelsep=55pt,radius=10pt,fillstyle=solid}
  \pstree[treemode=D]{\Tcircle[fillcolor=wheat]{ 1 }~[tnpos=l]{\shortstack[r]{\texttt{\detokenize{V1}}\\$\leq$\texttt{ 0.6464}}}
 }{
    \Tcircle[fillcolor=yellow]{ 2 }~{\makebox[\width]{\shortstack[c]{\emph{142}\\ 0.13 \\ \texttt{\detokenize{+U1}}}}}
  \pstree[treemode=D]{\Tcircle{ 3 }~[tnpos=l]{\shortstack[r]{\texttt{\detokenize{V2}}\\$\leq$\texttt{-0.4732}}}
 }{
    \Tcircle[fillcolor=purple]{ 6 }~{\makebox[\width]{\shortstack[c]{\emph{32}\\-1.11 \\ \texttt{\detokenize{+U2}}}}}
  \pstree[treemode=D]{\Tcircle{ 7 }~[tnpos=l]{\shortstack[r]{\texttt{\detokenize{value__median}}\\$\leq$\texttt{1.05}}}
 }{
    \Tcircle[fillcolor=yellow]{\small 14}~{\makebox[\width]{\shortstack[c]{\emph{76}\\ 0.20 \\ \texttt{\detokenize{+value__mean*}}}}}
  \pstree[treemode=D]{\Tcircle[linestyle=dashed]{\small 15}~[tnpos=l]{\shortstack[r]{\texttt{\detokenize{V3}}\\$\leq$\texttt{1.78}}}
 }{
    \Tcircle[fillcolor=yellow]{\small 30}~{\makebox[\width]{\shortstack[c]{\emph{25}\\ 0.88 \\ \texttt{\detokenize{+U3}}}}}
    \Tcircle[fillcolor=yellow]{\small 31}~{\makebox[\width]{\shortstack[c]{\emph{10}\\ 0.84 \\ \texttt{\detokenize{+U4}}}}}
 }
 }
 }
 }
 \end{center}
GUIDE v.39.0 0.250-SE
piecewise simple linear least-squares regression tree
 (constant fitted to incomplete cases)
for predicting \texttt{\detokenize{y}}.
 %Tree constructed with 285 observations.
 %Maximum number of split levels is 10 and minimum node sample size is 3.
At each split, an observation goes to the left branch 
 if and only if the condition is satisfied.
\texttt{\detokenize{V1}} = \texttt{\detokenize{value__root_mean_square}}.
\texttt{\detokenize{V2}} = \texttt{\detokenize{value__maximum}}.
\texttt{\detokenize{V3}} = \texttt{\detokenize{value__maximum}}.
 Circles with dashed lines are nodes with no significant split variables.
 Intermediate nodes with splits due to interaction are in wheat color.
Sample size (in \emph{italics}), mean of \texttt{\detokenize{y}}, and signed name of regressor variable printed below nodes.
 Terminal nodes with means above and below value of  0.099  at root node are colored yellow and purple respectively.
 Asterisk appended to regressor name indicates its slope is significant at the 0.05 level (unadjusted for multiplicity and model fitting).
 Second best split variable (based on interaction test) at root node is \texttt{\detokenize{value__median}}.
 %\end{landscape}
 \end{document}
