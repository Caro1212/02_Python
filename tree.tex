 \documentclass[12pt]{article}
 % LaTeX file created by GUIDE version 39.0 on 02/18/22 at 11:16
 %%%% Usage: latex tree.tex, dvips tree.tex, ps2pdf tree.tex
 %%%% or: xelatex tree.tex
 \usepackage{pstricks}
 %%%% If using pdflatex, replace above line with \usepackage[pdf]{pstricks}
 %%%% and use: pdflatex -shell-escape tree.tex
 \usepackage{pstricks,pst-node,pst-tree}
 \usepackage{geometry}
 \usepackage{lscape}
 \definecolor{wheat}{rgb}{0.96, 0.87, 0.7}
 \definecolor{lightgray}{rgb}{0.9, 0.9, 0.9}
 \definecolor{aqua}{rgb}{0.0, 1.0, 1.0}
 \definecolor{mauve}{rgb}{0.88, 0.69, 1.0}
 \definecolor{skyblue}{RGB}{86,180,233}
 \definecolor{vermillion}{RGB}{213,94,0}
 \definecolor{purple}{RGB}{204,121,167}
 \definecolor{bluishgreen}{RGB}{0,158,115}
 \pagestyle{empty}
 \begin{document}
 %\begin{landscape}
 \begin{center}
\psset{linecolor=black,tnsep=1pt,tndepth=0cm,tnheight=0cm,treesep=1.5cm,levelsep=55pt,radius=10pt,fillstyle=solid}
  \pstree[treemode=D]{\Tcircle{ 1 }~[tnpos=l]{\shortstack[r]{\texttt{\detokenize{y_0}}\\$\leq$\texttt{ 0.0591}}}
 }{
    \Tcircle[fillcolor=purple]{ 2 }~{\makebox[\width]{\shortstack[c]{\emph{119}\\-0.35 \\ \texttt{\detokenize{y_8}}}}}
  \pstree[treemode=D]{\Tcircle{ 3 }~[tnpos=l]{\shortstack[r]{\texttt{\detokenize{u_8}}\\$\leq$\texttt{-0.2220}}}
 }{
    \Tcircle[fillcolor=purple]{ 6 }~{\makebox[\width]{\shortstack[c]{\emph{32}\\ 0.01 \\ \texttt{\detokenize{y_8}}}}}
    \Tcircle[fillcolor=yellow]{ 7 }~{\makebox[\width]{\shortstack[c]{\emph{134}\\ 0.51 \\ \texttt{\detokenize{y_8}}}}}
 }
 }
 \end{center}
GUIDE v.39.0 0.250-SE
 piecewise polynomial least-squares regression tree
 of degree 2
 (constant fitted to incomplete cases)
for predicting \texttt{\detokenize{y_9}}.
 %Tree constructed with 285 observations.
 %Maximum number of split levels is 10 and minimum node sample size is 4.
At each split, an observation goes to the left branch 
 if and only if the condition is satisfied.
Sample size (in \emph{italics}), mean of \texttt{\detokenize{y_9}}, and name of regressor variable printed below nodes.
 Terminal nodes with means above and below value of  0.099  at root node are colored yellow and purple respectively.
 Second best split variable at root node is \texttt{\detokenize{y_1}}.
 %\end{landscape}
 \end{document}
