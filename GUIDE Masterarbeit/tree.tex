 \documentclass{article}
 % LaTeX file created by GUIDE version 39.0 on 06/07/22 at 10:01
 %%%% Usage: latex tree.tex, dvips tree.tex, ps2pdf tree.tex
 %%%% or: xelatex tree.tex
 \usepackage{pstricks}
 %%%% If using pdflatex, replace above line with \usepackage[pdf]{pstricks}
 %%%% and use: pdflatex -shell-escape tree.tex
 \usepackage{pstricks,pst-node,pst-tree}
 \usepackage[left=1in,right=1in,top=1in,bottom=1in]{geometry}
 \usepackage{lscape}
 \definecolor{wheat}{rgb}{0.96, 0.87, 0.7}
 \definecolor{lightgray}{rgb}{0.9, 0.9, 0.9}
 \definecolor{aqua}{rgb}{0.0, 1.0, 1.0}
 \definecolor{mauve}{rgb}{0.88, 0.69, 1.0}
 \definecolor{skyblue}{RGB}{86,180,233}
 \definecolor{vermillion}{RGB}{213,94,0}
 \definecolor{purple}{RGB}{204,121,167}
 \definecolor{bluishgreen}{RGB}{0,158,115}
 \pagestyle{empty}
 \begin{document}
 \begin{landscape}
 \begin{center}
\psset{linecolor=black,tnsep=1pt,tndepth=0cm,tnheight=0cm,treesep=1.0cm,levelsep=55pt,radius=10pt,fillstyle=solid}
  \pstree[treemode=D]{\Tcircle[linestyle=dashed]{ 1 }~[tnpos=l]{\shortstack[r]{\texttt{\detokenize{V1}}\\$\leq$\texttt{8871.4}}}
 }{
  \pstree[treemode=D]{\Tcircle[linestyle=dashed]{ 2 }~[tnpos=l]{\shortstack[r]{\texttt{\detokenize{sys__median}}\\$\leq$\texttt{143.17}}}
 }{
  \pstree[treemode=D]{\Tcircle[linestyle=dashed]{ 4 }~[tnpos=l]{\shortstack[r]{\texttt{\detokenize{bmi__median}}\\$\leq$\texttt{28.67}}}
 }{
    \Tcircle[fillcolor=purple]{ 8 }~{\shortstack[c]{\emph{15}\\166.9}}  
    \Tcircle[fillcolor=yellow]{ 9 }~{\shortstack[c]{\emph{7}\\215.1}}  
   }
    \Tcircle[fillcolor=yellow]{ 5 }~{\shortstack[c]{\emph{22}\\196.9}}  
   }
  \pstree[treemode=D]{\Tcircle{ 3 }~[tnpos=l]{\shortstack[r]{\texttt{\detokenize{V2}}\\$\leq$\texttt{9835.7}}}
 }{
    \Tcircle[fillcolor=purple]{ 6 }~{\shortstack[c]{\emph{18}\\166.2}}  
  \pstree[treemode=D]{\Tcircle{ 7 }~[tnpos=l]{\shortstack[r]{\texttt{\detokenize{bmi__minimum}}\\$\leq$\texttt{27.76}}}
 }{
  \pstree[treemode=D]{\Tcircle[linestyle=dashed]{\small 14}~[tnpos=l]{\shortstack[r]{\texttt{\detokenize{V3}}\\$\leq$\texttt{515.46}}}
 }{
    \Tcircle[fillcolor=purple]{\small 28}~{\shortstack[c]{\emph{21}\\166.2}}  
    \Tcircle[fillcolor=purple]{\small 29}~{\shortstack[c]{\emph{24}\\154.2}}  
   }
    \Tcircle[fillcolor=yellow]{\small 15}~{\shortstack[c]{\emph{22}\\199.0}}  
 }
 }
 }
 \end{center}
GUIDE v.39.0 
piecewise stepwise linear least-squares regression tree
 (missing regressor values imputed)
for predicting \texttt{\detokenize{y}}.
 %Tree constructed with 129 observations.
 Largest pruned tree with no more than 7 terminal nodes.
 %Maximum number of split levels is 5 and minimum node sample size is 5.
At each split, an observation goes to the left branch 
 if and only if the condition is satisfied.
\texttt{\detokenize{V1}} = \texttt{\detokenize{Insulin__normal_(human)__minimum}}.
\texttt{\detokenize{V2}} = \texttt{\detokenize{Insulin__normal_(human)__minimum}}.
\texttt{\detokenize{V3}} = \texttt{\detokenize{gewicht__sum_values}}.
 Circles with dashed lines are nodes with no significant split variables.
Sample size (in \emph{italics}) and mean of \texttt{\detokenize{y}} printed below nodes.
 Terminal nodes with means above and below value of 177.5 at root node are colored yellow and purple respectively.
 Second best split variable at root node is \texttt{\detokenize{Insulin-Isophan_(human)__mean}}.
 \end{landscape}
 \end{document}
